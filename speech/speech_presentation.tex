\documentclass[12pt, aspectratio=169]{beamer}
\usepackage[utf8]{inputenc}
\usepackage[T2A]{fontenc}
\usepackage[english, russian]{babel}
\usepackage{geometry}
\usepackage{multirow, marginnote, hyperref, graphicx, import, enumitem, array}
\graphicspath{{./images/}}
\newcommand{\task}[2]{\textbf{(#1)} #2}

\usepackage[inline]{asymptote}
\begin{asydef}
    settings.outformat = "pdf";
    import geometry;
\end{asydef}

\hypersetup{
    colorlinks=true,
    linkcolor=green!60!black,
    urlcolor=blue,
    pdftitle={Презентация факультатива "Утром на ушку, а вечером строить перпендикуляры"}
    pdfauthor={Лунёв Егор}
}

\setlength{\tabcolsep}{8pt}
\renewcommand{\arraystretch}{1}
\setlength{\arrayrulewidth}{0.3mm}


\usetheme{Madrid}
\usecolortheme{seahorse}
\usefonttheme{default}
\setbeamertemplate{navigation symbols}{}
\setbeamertemplate{headline}{}
\setbeamercovered{transparent}

\title[Перпендикулярные конструкции]{Утром на ушку, а вечером строить перпендикуляры}
\subtitle{Факультатив}
\date[КЛШ]{14.04.2024}
\author{Лунёв Егор (\href{https://t.me/egorrshuk}{\texttt{@egorrshuk}})} 
\logo{\includegraphics[height=1.3cm]{Wrom_dtERac.jpg}}

\begin{document}

\maketitle

\section{Основная информация}
\begin{frame}{Q\&A}
    \begin{columns}
        \column{0.5\linewidth}
            \begin{itemize}
                \item<1> Какая тема моего факультатива?
                \item<2> Какими минимальными знаниями должен обладать школьник?
                \item<3> Школьников каких классов я жду?
                \item<4> Какое количество школьников?
                \item<5> Что мне нужно для занятия?
            \end{itemize}
        \column{0.5\linewidth}
            \only<1>{\textbf{Тема} моего факультатива -- это изучение конструкций, с перпендикулярными прямыми, ев\-кли\-до\-вой \textbf{планиметрии},  применимые в решении \textbf{олимпиадных} задач.}
            \only<2>{\textbf{Threshold}'ом факультатива являются знания о вписанных четырехугольниках.}
            \only<3>{Мой факультатив сможет понять любой, нужно лишь знать \textbf{минимум}.}
            \only<4>{Я жду до \textbf{8-ми} школьников.}
            \only<5>{Доска \emph{(маркерная)} и маркеры. Возможно распечатать материалы. Возможно проектор.}
    \end{columns}
\end{frame}

\begin{frame}{Аннотация}
    \import{./}{abstract.tex}
\end{frame}

\begin{frame}{Общий план занятий}
    \begin{table}[ht]
        \centering
        \begin{tabular}{|r|p{0.2\linewidth}|p{0.3\linewidth}|p{0.2\linewidth}|}
            \hline
            
            \textbf{№} & \textbf{Темы} & \textbf{Теоретический материал и чепуха} & \textbf{Практический материал} \\
            
            \hline
            
            1 & Счет углов & Методы счета, вписанные четырехугольники, полезная (!) лемма & \multirow{3}{\linewidth}{Решение задач, помощь в решении задач, разбор задач (по возможности)} \\ \cline{1-3}
            2 & Свойства ортоцентра и окружность Эйлера & Симметрии ортоцентра и другие свойства, окружность Эйлера и прямая Эйлера  & \\ \cline{1-3}
            3 & О\-рто\-ди\-а\-го\-на\-ль\-ны\-е четырехугольники & Теорема об ортодиагональных четырехугольниках в отрезках &  \\ \hline
        \end{tabular}
    \end{table}
\end{frame}
\addtocounter{framenumber}{-1}

\begin{frame}{Общий план занятий}
    \begin{table}[ht]
        \centering
        \begin{tabular}{|r|p{0.2\linewidth}|p{0.3\linewidth}|p{0.22\linewidth}|}
            \hline
            \textbf{№} & \textbf{Темы} & \textbf{Теоретический материал и чепуха} & \textbf{Практический материал} \\ \hline
            4 & Радикальная ось и линия центров & Степень точки, радикальная ось и радикальный центр, их частные случаи. Лемма об ортогональных четырехугольниках & \multirow{3}{\linewidth}{Решение задач, помощь в решении задач, разбор задач (по возможности)} \\ \cline{1-3}  
            5 & Прямая Симсона & Прямая Симсона и ссылка на точку Микеля &  \\ \cline{1-3}
            6 & Задача 255 & Лемма 255 и ее внешние случаи. Внешние случаи конструкций с биссектрисами &  \\ \hline
        \end{tabular}
    \end{table}
\end{frame}

\begin{frame}{Пример средней задачи}
    \begin{columns}
        \column{0.37\textwidth} 
        \begin{block}{Задача 6b}
        \task{Точка Микеля четырехсторонника}{На плоскости даны четыре прямые общего положения. Эти прямые образуют $4$ треугольника. Докажите, что описанные окружности этих треугольников пересекаются в одной точке.} 
        \end{block}
        \column{0.55\textwidth}
        \textbf{Решение:}
        
        Пусть на первой прямой лежат точки $A$, $F$ и $B$, на второй $B$, $D$ и $C$, на третьей $C$, $A$ и $E$ и на четвертой $E$, $D$ и $F$. Тогда по задаче {\color{green!60!black}6a} для $\triangle ABC$ и точек $F$, $D$ и $E$:
        \begin{equation}
            (AFE) \cap (BFD) \cap (CDE) = M. \label{eq:th:miquel's point 1}
        \end{equation}

        По задаче {\color{green!60!black}6a} для $\triangle AFE$ и точек $B$, $D$ и $C$:
        \begin{equation}
            (ABC) \cap (FBD) \cap (EDC) = G. \label{eq:th:miquel's point 2}
        \end{equation}
        
        Но по утверждениям (\ref{eq:th:miquel's point 1}) и (\ref{eq:th:miquel's point 2}): $G \equiv M$. Отсюда следует, что все нужные окружности пересекаются в одной точке.
    \end{columns}
\end{frame}

\begin{frame}{План 6-го занятия}
    \centering
    \begin{tabular}{c | p{0.7\linewidth}}                
        3 минуты & Рассказ о появлении этой задачи, о Игоре Федоровиче Шарыгине, о олимпиаде его имени. \\ [3mm]

        5 минут & Доказательство самой задачи, ссылка на прямую Симсона. \\ [3mm]
        
        5 минут & Внешний случай теоремы. Разговор о внешних случаях для биссектрис. \\ [3mm]
        
        27 минут & Решение задач, помощь школьникам с решением.\\ [3mm]
        
        \textit{10 минут} & Разбор задач по надобности школьников, или по моему желанию. Время которое не потрачено на разбор, будет потрачено на решение задач :)\\ [3mm]
        
        \hline 
        \textbf{50 минут} & \textbf{Конец занятия}
    \end{tabular}
\end{frame}
\end{document}


