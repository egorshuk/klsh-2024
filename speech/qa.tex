\begin{description}
    \item[На кого рассчитан факультатив?] Мой факультатив рассчитан на школьников, которые уже знают, что такое вписанные четырехугольники, это и является минимальными знаниями для факультатива. Поэтому на мой факультатив может пойти, как 8-, так и 9- или 10-классник.

    \item[Какое количество школьников я жду?] Я жду у себя до 8-ми человек.

    \item[В каком формате будет проходить занятие?] Вначале занятия я буду рассказывать теоретический материал -- доказывать какие-то базовые или важные свойства, теоремы, леммы. После этого я буду давать им какое-то количество задач на эту тему. По ходу решения этих заданий, я буду помогатьв их решении, слушать их решения. В конце занятия будет разбор заданий, которые просят школьники или которые кажутся мне важными или интересными.

    \item[Какое количество задач будет?] Для каждого занятия я старался оценить сверху время для решения задач, чтобы понять минимум задач на урок, которые я должен предоставить. Если я буду не успевать что-то говорить, рассказывать, то это время будет отниматься именно у решения задач.

    С расчетом, что средний школьник в среднем решает среднюю задачу не менее чем за 3 минуты \textit(опять стремление к оценки сверху количества, чтобы точно хватило), то нужно минимум 7-8 задач, но будет больше (хотелось бы минимум 10 на каждую тему, на каждое занятие).
    
    \item[Если школьники будут слабые/сильные?] Если ко мне придут школьники, которые мало образованные в теме, для которых материал будет слишком сложный. То я уберу темы: <<прямая Симсона>> и <<радикальная ось и линия центров>>. Тогда темы <<свойства ортоцентра и окружность Эйлера>> и <<ортодиагональные четырехугольники>> растянутся на два урока каждая. 
    
    Если же ко мне прийдут слишком сильные школьники, которые будут щелкать задачи, то я там им <<Геометрию в картинках>> А.В. Акопяна, там много задач, которые их удовлетворят. Но такая ситуация нежелательна, я им не дам особо новых тем, а только задач...

    \item[Что мне нужно для занятия?] Из оборудования на всех занятиях мне понадобится доска, желательно маркерная, и сами маркеры. Также мне может понадобиться напечатать материалы для факультатива. Еще не слишком принципиально, но мне мог бы понадобиться проектор, чтобы показывать чертежи некоторых задач в Geogebra.

    \item[Чему научатся школьники?] Они научаться применять классические методы решения задач, такие как: счет углов, счет отрезков и нахождение базовых конструкций. А также изучат эти самые базовые конструкции.

    \item[Будет ли что-то на ВИП?] В самих буклетах есть глава ''Заметки``, туда школьники могут писать что хотят. Но я планирую, чтобы писали задачи, чертили задачи, придумывали задачи. И у меня есть такая идея: по ходу сезона у школьника появляется любимая задача (может она была очень сложной для него, а может просто показалась какой-то красивой). И он методами этого факультатива чертит и описывает ее доказательство, на каком-нибудь плакате. Будет замечательно, если он сможет придумать какое-то обобщение для задачи!
\end{description}
\pagebreak