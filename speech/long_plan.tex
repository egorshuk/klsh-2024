\subsection{Счет углов}
\begin{center}
    \begin{tabular}{c | p{0.7\linewidth}}
        5 минут & Рассказ о себе, откуда я, кто я. Для кого предназначен факультатив, и чего я от них жду. Спрошу кто чего уже знает, чего ждет от факультатива и зачем пришел. \\ [3mm]
        
        5 минут & Рассказ о канале <<Олимпиадная геометрия>>, рассказ о стиле иллюстраций, о книге <<Геометрия в картинках>>  А.В. Акопяна, раздача брошюр. \\ [3mm]
        
        5 минут & Рассказ о счете углов, о лемме {\color{red}1.2} и доказательство существования ортоцентра. \\ [3mm]
        
        25 минут & Решение задач, помощь школьникам с решением.\\ [3mm]
        
        \textit{10 минут} & Разбор задач по надобности школьников, или по моему желанию. Время которое не потрачено на разбор, будет потрачено на решение задач :)\\ [3mm]
        
        \hline 
        \textbf{50 минут} & \textbf{Конец занятия}
    \end{tabular}
\end{center}

\subsection{Свойства ортоцентра и окружность Эйлера}
\begin{center}
    \begin{tabular}{c | p{0.7\linewidth}}                
        15 минут& Рассказ об основных свойствах ортоцентра\footnotemark(по крайней мере о тех, которые нужны для доказательство окружности Эйлера). Рассмотрение базовой картины с параллелограммом. Доказательство окружности Эйлера.\\ [3mm]
        
        25 минут & Решение задач, помощь школьникам с решением.\\ [3mm]
        
        \textit{10 минут} & Разбор задач по надобности школьников, или по моему желанию. Время которое не потрачено на разбор, будет потрачено на решение задач :)\\ [3mm]
        
        \hline 
        \textbf{50 минут} & \textbf{Конец занятия}
    \end{tabular}
\end{center}
\footnotetext{Какие-то факты планируется дать на самостоятельное рассмотрение.}
\subsection{Ортодиагональные четырёхугольники}
\begin{center}
    \begin{tabular}{c | p{0.7\linewidth}}                
        5 минут & Простецкое и супербыстрое доказательство этого свойства \\ [3mm]
        
        5 минут & Небольшой разговор о том, что вот возникают отрезки и это по-сути уже счет в отрезках. Что это вообще такое?? \\ [3mm]
        
        30 минут & Решение задач, помощь школьникам с решением.\\ [3mm]
        
        \textit{10 минут} & Разбор задач по надобности школьников, или по моему желанию. Время которое не потрачено на разбор, будет потрачено на решение задач :)\\ [3mm]
        
        \hline 
        \textbf{50 минут} & \textbf{Конец занятия}
    \end{tabular}
\end{center}
\subsection{Радикальная ось и линия центров}
\begin{center}
    \begin{tabular}{c | p{0.7\linewidth}}                
        5 минут & Продолжение разговора о счете в отрезках, рассмотрение понятия о степени точки, доказательство всех свойств степени точки. \\ [3mm]

        10 минут & Рассказ о радикальной оси, радикальном центре, о важности. Доказательство всех свойств. \\ [3mm]
        
        25 минут & Решение задач, помощь школьникам с решением.\\ [3mm]
        
        \textit{10 минут} & Разбор задач по надобности школьников, или по моему желанию. Время которое не потрачено на разбор, будет потрачено на решение задач :)\\ [3mm]
        
        \hline
        \textbf{50 минут} & \textbf{Конец занятия}
    \end{tabular}
\end{center}

\subsection{Прямая Симсона}
\begin{center}
    \begin{tabular}{c | p{0.7\linewidth}}                
        3 минуты & История о Уильяме Уоллесе, несправедливости и теореме Арнольда. \\ [3mm] 
        7 минут & Доказательство прямой Симсона, ссылка на точку Микеля. \\ [3mm]
        
        30 минут & Решение задач, помощь школьникам с решением.\\ [3mm]
        
        \textit{10 минут} & Разбор задач по надобности школьников, или по моему желанию. Время которое не потрачено на разбор, будет потрачено на решение задач :)\\ [3mm]
        
        \hline 
        \textbf{50 минут} & \textbf{Конец занятия}
    \end{tabular}
\end{center}

\subsection{Задача 255}
\begin{center}
    \begin{tabular}{c | p{0.7\linewidth}}                
        3 минуты & Рассказ о появлении этой задачи, о Игоре Федоровиче Шарыгине, о олимпиаде его имени. \\ [3mm]

        5 минут & Доказательство самой задачи, ссылка на прямую Симсона. \\ [3mm]
        
        5 минут & Внешний случай теоремы. Разговор о внешних случаях для биссектрис. \\ [3mm]
        
        27 минут & Решение задач, помощь школьникам с решением.\\ [3mm]
        
        \textit{10 минут} & Разбор задач по надобности школьников, или по моему желанию. Время которое не потрачено на разбор, будет потрачено на решение задач :)\\ [3mm]
        
        \hline 
        \textbf{50 минут} & \textbf{Конец занятия}
    \end{tabular}
\end{center}