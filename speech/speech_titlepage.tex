\begin{titlepage}
    \begin{center}
    \LARGE
    %\textbf{Хоровод вокруг перпендикулярных конструкций}
    Паспорт факультатива <<Утром на ушку, а вечером строить перпендикуляры>>
    
    \vspace{5mm}
    
    \Large
    Направление точных наук
    
    \vspace{5mm}

    Лунёв Егор (\href{https://t.me/egorrshuk}{\texttt{@egorrshuk}})

    \vspace{2cm}

    \begin{figure}[h]
        \centering
        \begin{asy}
            size(16cm);
            triangle tri = triangleabc(8, 7, 8.5);
            point H = orthocenter(tri); 
    
            line t = line(H, 27); draw(t, red); line l = line(H, 117); draw(l, orange);
    
            point T_a = intersectionpoint(t, tri.BC);
            point T_b = intersectionpoint(t, tri.AC); 
            point T_c = intersectionpoint(t, tri.BA); 
    
            point L_a = intersectionpoint(l, tri.BC); 
            point L_b = intersectionpoint(l, tri.AC); 
            point L_c = intersectionpoint(l, tri.BA); 
    
            draw(line(tri.VA, T_c));
            draw(line(tri.VC, L_a));
            draw(segment(tri.VA, tri.VC));
            
            point M_a = midpoint(segment(L_a, T_a));
            point M_b = midpoint(segment(L_b, T_b));
            point M_c = midpoint(segment(L_c, T_c));
    
            draw(segment(L_a, T_a), cyan, StickIntervalMarker(2, 2, size=8));
            draw(segment(L_b, T_b), cyan, StickIntervalMarker(2, 1, size=6));
            draw(segment(L_c, T_c), cyan, StickIntervalMarker(2, 3, size=8));
    
            perpendicularmark(l, t, deepgreen, size=10, quarter=3);
    
            draw(line(M_a, M_c), dashed+brown);
    
            dot(T_a, deepgreen); dot(T_b, deepgreen); dot(T_c, deepgreen);
            dot(L_a, brown); dot(L_b, brown); dot(L_c, brown);
            dot(M_a); dot(M_b); dot(M_c);
            dot(H, 5+deepgreen);
    
            draw(box((-4, -1), (9.5, 12.5)), invisible);
        \end{asy}
    \end{figure}

    \vspace{1cm}
    \vfill

    \large
    Красноярск (не Орбита), 2024
    \end{center}
\end{titlepage}