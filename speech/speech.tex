\documentclass[12pt, oneside]{article}
\usepackage{ifxetex}
\ifxetex
    \usepackage{xltxtra}
    \defaultfontfeatures{Ligatures=TeX,Mapping=tex-text}

    \setmainfont{Brygada1918}[
    Path=../fonts/Brygada1918/,
    Extension = .ttf,
    UprightFont=*-Regular,
    BoldFont=*-Bold,
    ItalicFont=*-Italic,
    BoldItalicFont=*-BoldItalic
    ]
    
    \setsansfont{Montserrat}[
    Path=../fonts/Montserrat/,
    Scale=0.9,
    Extension = .ttf,
    UprightFont=*-Medium,
    BoldFont=*-Bold,
    ItalicFont=*-MediumItalic,
    BoldItalicFont=*-BoldItalic
    ]
    
    %\setmonofont{CascadiaCode}[
    %Path=./fonts/Cascadia/,
    %Scale=0.85,
    %Extension = .ttf,
    %UprightFont=*-Regular,
    %BoldFont=*-Bold,
    %ItalicFont=*-Italic,
    %BoldItalicFont=*-BoldItalic
    %]

    \setmonofont{Iosevka}[
        Path=./fonts/Iosevka/,
        Scale=0.9,
        Extension = .ttc,
        UprightFont=*-Regular,
        BoldFont=*-Bold
    ]
    
    \input{tucyradd.def}
\else
    \usepackage[utf8]{inputenc}
    \usepackage[T2A]{fontenc}
\fi
\usepackage[english, russian]{babel}
\usepackage{geometry}
\usepackage{multirow, marginnote, hyperref, graphicx, import, enumitem}
\usepackage[russian, noabbrev]{cleveref}
\graphicspath{{../images/}}
\geometry{
    a4paper,
    margin = 2cm,
    footnotesep = 6mm
}
\usepackage[table]{xcolor}
\usepackage[inline]{asymptote}
\begin{asydef}
    settings.outformat = "pdf";
    import geometry;
\end{asydef}

\hypersetup{
    colorlinks=true,
    linkcolor=red,
    urlcolor=blue,
    pdftitle={Паспорт факультатива "Утром на ушку, а вечером строить перпендикуляры"}
    pdfauthor={Лунёв Егор}
}

\setlength{\tabcolsep}{10pt}
\renewcommand{\arraystretch}{1.5}
\setlength{\arrayrulewidth}{0.3mm}
\arrayrulecolor{blue!50!black}

\begin{document}

\thispagestyle{empty}
\import{speech/}{speech_titlepage.tex} 
\pagebreak

\thispagestyle{empty}
\import{./}{abstract.tex}
\begin{figure}[h]
    \centering
    \includegraphics[width=0.6\textwidth]{chelik.png}
    \label{fig:abstract}
\end{figure}
\vfill
\pagebreak

\setcounter{page}{1}
\section{Основная информация}
%\begin{center}
%    \begin{tabular}{p{0.25\linewidth}|p{0.7\linewidth}}
%        Тема факультатива & Изучение конструкций, с перпендикулярными прямыми, ев\-кли\-до\-вой геометрии,  применимые в решении олимпиадных задач \\
%        Цель факультатива & Научить школьников применять конструкции и методы, представленным в факультативе \\
%    \end{tabular}
%\end{center}


\subsection{Q\&A}

\import{./}{qa.tex}

\subsection{Общий план занятий}

\import{./}{short_plan.tex}


\section{Планы занятий по отдельности}
Опять стремление к \textbf{максимизации}. Только теперь максимизация \textbf{времени решения задач}, для этого я везде давал \textbf{оценку сверху} времени, которое я буду тратить на теорию. Если я буду рассказывать ее быстрее -- замечательно, значит будем дольше решать задачи. 

Для разбора я выделил 10 минут. Этого времени, как мне кажется, должно хватить -- там ведь самые интересные задачи будут, а их не очень много. Если же школьникам потребуется больше времени, то это не проблема -- задачу ведь можно рассказать и вне факультативного времени, если школьник заинтересован...

Также я провел все занятия этого факультатива со средними школьниками, у них были какие-то знаний по предмету, но не было знаний никаких знаний, о какой-либо из рассматриваемых тем. И по результатам этого занятий могу сделать такие выводы:

\begin{enumerate}
    \item Задач очень много, в основном школьники успевают решать 4-6 задач за урок.
    \item Для разбора хватает времени, т.к. разбирать первые задачи (простые) -- легко и быстро.
    \item Для рассказывания теоретического материала тоже хватает времени, даже если не сильно торопиться.
    \item Очень удобно писать ключевые задачи для урока, чтобы школьники успевали брать самые важные знания.
    \item Интеракции со школьниками должно быть больше: нужно чаще говорить с ними (как о задачах, так может и простой какой-то small talk). Как только он решил задачу -- слушать их.
\end{enumerate}

\import{./}{long_plan.tex}

\end{document}