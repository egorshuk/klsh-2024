\begin{titlepage}
    \begin{center}
    
    \LARGE
    %\textbf{Хоровод вокруг перпендикулярных конструкций}
    %\textbf{Утром на ушку, а вечером строить перпендикуляры}
    \textbf{Топ-5 теорем, которые не помогут построить дом}


    \Large 
    \if\magic S{(с решениями)}\else{}\fi

    \vspace{5mm}
    
    Направление точных наук
    
    \vspace{5mm}
    
    Лунёв Егор (\href{https://t.me/egorrshuk}{\texttt{@egorrshuk}})

    \vspace{1cm}

    \begin{figure}[h]
        \centering
        \begin{asy}
            size(12.3cm);
            triangle t=triangleabc(18.5, 15, 19.5);
            draw(t, linewidth(1.4*bp));
            point P = 0.47*t.B+0.3*(t.C-t.B); dot(P, orange+4);
            point Q = isogonalconjugate(t, P); dot(Q, brown+4);
            
            point P_a = projection(t.BC)*P; segment p_a = segment(P, P_a); draw(p_a, grey); //dot(P_a, royalblue);
            point P_b = projection(t.AC)*P; segment p_b = segment(P, P_b); draw(p_b, grey); //dot(P_b, royalblue);
            point P_c = projection(t.AB)*P; segment p_c = segment(P, P_c); draw(p_c, grey); //dot(P_c, royalblue);
            
            point Q_a = projection(t.BC)*Q; segment q_a = segment(Q, Q_a); draw(q_a, grey); //dot(Q_a, deepgreen);
            point Q_b = projection(t.AC)*Q; segment q_b = segment(Q, Q_b); draw(q_b, grey); //dot(Q_b, deepgreen);
            point Q_c = projection(t.AB)*Q; segment q_c = segment(Q, Q_c); draw(q_c, grey); //dot(Q_c, deepgreen);

            perpendicularmark(t.AB, p_c, size=10, blue); perpendicularmark(t.AB, q_c, size=10, deepgreen);
            perpendicularmark(t.BC, p_a, size=10, blue); perpendicularmark(t.BC, q_a, size=10, deepgreen);
            perpendicularmark(reverse(t.AC), p_b, size=10, blue); perpendicularmark(reverse(t.AC), q_b, size=10, deepgreen);

            circle G = circle(P_a, P_b, P_c); draw(G, 1.1*bp+mediumred+dashed);

            segment pa = segment(t.VA, P); draw(pa, orange);
            segment pb = segment(t.VB, P); draw(pb, orange); 
            segment pc = segment(t.VC, P); draw(pc, orange);

            segment qa = segment(t.VA, Q); draw(qa, brown);
            segment qb = segment(t.VB, Q); draw(qb, brown); 
            segment qc = segment(t.VC, Q); draw(qc, brown);

            markangle(n = 1, line(pa), line(t.AC), radius=1.3cm, orange);
            markangle(n = 1, line(t.AB), line(qa), radius=1cm, brown);

            markangle(n = 2, rotate(180)*line(t.AC), line(pc), radius=1.3cm, orange);
            markangle(n = 2, line(qc), line(t.BC), radius=1cm, brown);

            markangle(n = 3, line(t.BC), line(qb), radius=1.3cm, brown);
            markangle(n = 3, line(pb), line(t.AB), radius=1.4cm, orange);

            triangle t_P = pedal(t, P); draw(t_P, royalblue);
            triangle t_Q = pedal(t, Q); draw(t_Q, deepgreen);

            //segment pq = segment(P, Q);
            //point M = midpoint(pq); dot(M, purple); draw(pq, purple, //StickIntervalMarker(2,1, size=10));

            //point M_a = projection(t.BC)*M; segment m_a = segment(M, M_a); //draw(m_a, grey); dot(M_a, purple);
            //point M_b = projection(t.AC)*M; segment m_b = segment(M, M_b); //draw(m_b, grey); dot(M_b, purple);
            //point M_c = projection(t.BA)*M; segment m_c = segment(M, M_c); //draw(m_c, grey); dot(M_c, purple);
            
            //perpendicularmark(t.AB, m_c, size=10, purple); //perpendicularmark(reverse(t.AC), m_b, size=10, purple);
            //perpendicularmark(t.BC, m_a, size=10, purple);

            //draw(circle(M_a, length(segment(P_a, Q_a)/2)), dashed + brown);
            //draw(circle(M_b, length(segment(P_b, Q_b)/2)), dashed + brown);
            //draw(circle(M_c, length(segment(P_c, Q_c)/2)), dashed + brown);
        \end{asy}
        \centering
        \label{fig:titlep_age}
    \end{figure}

    \vspace{1cm}
    \vfill

    \large
    Место под соснами
    
    Лето, 2024
    
    \end{center}
\end{titlepage}
