\begin{enumerate}[resume*]
    \item Докажите, что высоты треугольника конкурентны. \texttt{0\_0} \wishlisted
    \item Окружность делит каждую из сторон треугольника на три равные части. Докажите, что этот треугольник -- равносторонний. \wishlisted
    \item Окружности ${\color{red}{\psi}}$ и ${\color{blue}{\omega}}$ вписаны в вертикальный угол $\angle {\color{red}{n}}{\color{blue}{m}}$, ${\color{red}{\psi}}$ касается прямой ${\color{red}{n}}$ в точке ${\color{red}{N}}$, а ${\color{blue}{\omega}}$ касается прямой ${\color{blue}{m}}$ в точке ${\color{blue}{M}}$. Докажите, что ${\color{red}{\psi}}$ и ${\color{blue}{\omega}}$ высекают на ${\color{red}{N}}{\color{blue}{M}}$ равные отрезки. \wishlisted
    \item \task{ММО, 2013, 11.3}{Четырёхугольник $ABCD$ такой, что $AB = BC$ и $AD = DC$. Точки $K$, $L$ и $M$ -- середины отрезков $AB$, $CD$ и $AC$ соответственно. Перпендикуляр, проведённый из точки $A$ к прямой $BC$, пересекается с перпендикуляром, проведённым из точки $C$ к прямой $AD$, в точке $T$. Докажите, что прямые $KL \perp TM$.} \wishlisted
    \item Точка $D$ -- основание биссектрисы из точки $A$ треугольника $ABC$. Окружность $(ABD)$ повторно пересекает прямую $AC$ в точке $E$, а окружность $(ACD)$ повторно пересекает прямую $BC$ в точке $F$. Докажите, что $BF = CE$. \wishlisted
    \item Окружность $\omega$ проходит через вершины $A$ и $D$ равнобокой трапеции $ABCD$ так, что пересекает диагональ $BD$ и боковую сторону $CD$ в точках $P$ и $Q$ соответственно. Точки $P'$ и $Q'$ симметричны точкам $P$ и $Q$ относительно середин отрезков $BD$ и $CD$ соответственно. Докажите, что $B$, $C$, $P'$ и $Q'$ концикличны. \wishlisted
    \item \task{JBMO Shortlist, 2022, G6}{Пусть $\Omega$ -- описанная окружность треугольника $ABC$. Взяты точки $P$ и $Q$, так что $P$ равноудалена от $A$ и $B$, а $Q$ равноудалена от $A$ и $C$ и углы $PBC$ и $QCB$ равны. Докажите, что касательная к $\Omega$ в точке $A$, прямая $PQ$ и $BC$ пересекаются в одной точке.} 
    \item Вневписанные окружности $\omega_b$ и $\omega_c$ треугольника $ABC$ касаются сторон $AC$ и $AB$ соответственно в точках $E$ и $F$. Прямая $EF$ повторно пересекает окружности $\omega_b$ и $\omega_c$ в точках $X$ и $Y$ соответственно. Касательные в точках $X$ и $Y$ проведенные к окружностям $\omega_b$ и $\omega_c$ пересекают прямые $AC$ и $AB$ в точках $K$ и $L$ соответственно. Докажите, что середина отрезка $KL$ равноудалена от точек $E$ и $F$.
    \item \begin{enumerate}
        \item Пусть $C_1$ и $B_1$  -- точки на сторонах $AB$ и $AC$ треугольника $ABC$ соответственно. Докажите что, радикальная ось окружностей, построенных на $BB_1$ и $CC_1$ как на диаметре, проходит через ортоцентр треугольника $ABC$.
        \item \task{Ось Обера}{Докажите, что четыре ортоцентра четырёх треугольников, образованных четырьмя попарно пересекающимися прямыми, никакие три из которых не проходят через одну точку\footnote{Такие прямые образуют фигуру, называемую полным четырёхсторонником.}, коллинеарны.}
        \item \task{Теорема Гаусса-Боденмиллера}{Докажите, что прямая Гаусса\footnote{Прямой Гаусса полного четырёхсторонника называется прямая, проходящая через середины трех его диагоналей.} перпендикулярна оси Обера.}
    \end{enumerate}
    \item Чевианы $AD$, $BE$ и $CF$ треугольника $ABC$ конкурентны. Прямая $EF$ пересекает окружность $(ABC)$ в точках $P$ и $Q$. Докажите, что $P$, $Q$, $D$ и середина отрезка $BC$ концикличны.
    \item \task{Устная олимпиада по геометрии, 2014, 10-11.4}{Медианы $AM_a$, $BM_b$ и $CM_c$ остроугольного треугольника $ABC$ пересекаются в точке $G$, а высоты $AH_a$, $BH_b$ и $CH_c$ -- в точке $H$. Касательная к окружности девяти точек треугольника $ABC$ а в точке $H_c$ пересекает прямую $M_aM_b$ в точке $C'$. Точки $A'$ и $B'$ определяются аналогично. Докажите, что $A'$, $B'$ и $C'$ лежат на одной прямой, перпендикулярной прямой $GH$.}
    \item В треугольнике $ABC$ высоты $AD$, $BE$, $CF$ пересекаются в точке $H$. Прямые $DE$, $EF$ и $DF$ пересекаются прямые $AB$, $BC$ и $AC$. В точках $A_1$, $B_1$, $C_1$ соответственно. Докажите, что точки $A_1$, $B_1$, $C_1$ лежат на прямой\footnote{Такая прямая называется трилинейной полярой ортоцентра, или ортоцентрической осью, или центральной линией центра описанной окружности.} перпендикулярной прямой Эйлера треугольник $ABC$.
    %\item \task{ММО, 2013, 10.6}{Пусть $I$ -- инцентр неравнобедренного треугольника $ABC$. $A_1$ -- середина дуги $BC$ описанной окружности треугольника $ABC$, не содержащей точки $A$, а $A_2$ -- середина дуги $BAC$. Перпендикуляр, опущенный из точки $A_1$ на прямую $A_2I$, пересекает прямую $BC$ в точке $A'$. Аналогично определяются точки $B'$ и $C'$. \begin{enumerate}
        %\item Докажите, что точки $A'$, $B'$, $C'$ коллинеарны.
        %\item Докажите, что эта прямая перпендикулярна прямой $OI$%\footnote{Можно рассматривать степень точки относительно вырожденной окружности.}, где $O$ -- центр описанной окружности треугольника $ABC$.
    %\end{enumerate}}
\end{enumerate}