\begin{enumerate}[resume*]
    \item \task{ММО, 1994}{В треугольнике $ABC$ точки $M$ и $N$ -- проекции вершины $B$ биссектрисы углов $A$ и $C$, а $P$ и $Q$ -- проекции на внешние биссектрисы этих же углов.\begin{enumerate}
        \item Докажите, что точки $M$, $N$, $P$ и $Q$ коллинеарны. \wishlisted
        \item Докажите, что длина отрезка $PQ$ равна полупериметру треугольника $ABC$.
    \end{enumerate}}

    \item В трапецию $ABCD$ вписанная окружность с центром $I$. Окружность вписанная в треугольник $ACD$ касается сторон $AD$ и $AC$ в точках $E$ и $F$. Докажите, что точки $E$, $F$ и $I$ коллинеарны. \wishlisted
    
    \item \task{Ф. Л. Бахарев, Санкт-Петербургская о\-лим\-пи\-а\-да, 1999}{В неравнобедренном треугольнике $ABC$ проведены биссектрисы $AA_1$ и $CC_1$ и отмечены точки $K$ и $L$ -- се\-ре\-ди\-ны сторон $AB$ и $BC$ соответственно. $AP$ и $CQ$ -- перпендикуляры, опущенные на $CC_1$ и $AA_1$ соответственно. Докажите, что прямые $PK$ и $QL$ пересекаются на стороне $AC$.}
    \item В равнобедренном треугольнике $ABC$ $(AB = BC)$ средняя линия, параллельная стороне $BC$ пересекается со вписанной окружностью в точке $D$, не лежащей на $AC$. Докажите, что касательная к окружности в точке $D$ пересекается с биссектрисой угла $C$ на стороне $AB$. \wishlisted
    \item {\begin{enumerate}
        \item \task{Первая внешняя Лемма 255}{Пусть $M$ и $N$ -- точки касания вневписанной окружности $\omega_a$ треугольника $ABC$ со стороной $BC$ и продолжением стороны $AC$, а $P$ -- точка пересечения бис\-сек\-три\-сы угла $A$ c прямой $MN$. Докажите, что $\angle APB = 90^\circ$.} \wishlisted
        \item \task{Вторая внешняя Лемма 255}{Пусть $M$ и $N$ -- точки касания вневписанной окружности $\omega_a$ треугольника $ABC$ со продолжениями сторон $AB$ и $AC$, а $P$ -- точка пересечения бис\-сек\-три\-сы внешнего угла $B$ c прямой $MN$. Докажите, что $\angle BPC = 90^\circ$.} \wishlisted
    \end{enumerate}}
    \item В треугольнике $ABC$ точки $A_c$, $B_c$, $C_c$ -- точки касания прямых $BC$, $AC$ и $AB$ с вневписанной окружностью $\omega_c$ (с центром в $I_c$). Точки $A_b$, $B_b$, $C_b$ определяются аналогично.
    $$\begin{cases}
        B_1 \equiv A_cC_c \cap A_bC_b \\
        C_1 \equiv A_bB_b \cap A_cB_c \\
        A_1 \equiv A_bB_b \cap A_cC_c \\
        A_2 \equiv A_cB_c \cap A_bC_B
    \end{cases}.$$
    \begin{enumerate}
        \item Докажите, что точки $A$, $B_1$, $C_1$, $I_b$, $I_c$ коллинеарны.
        \item Докажите, что \(A_1A_2 \perp BC\).
    \end{enumerate}
\end{enumerate}