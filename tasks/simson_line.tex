\begin{enumerate}[resume*]
    \item Дан прямоугольник $ABCD$. Через точку $B$ провели две перпендикулярные прямые. Первая прямая пересекает сторону $AD$ в точке $K$, а вторая — продолжение стороны $CD$ в точке $L$. $F$ -- точка пересечения $KL$ и $AC$. Докажите, что $BF \perp KL$. \wishlisted

        \solution{
            Точка $B$ лежит на окружности $(DKL)$, а также $BA \perp KD$ и $BC \perp DL$, значит $AC$ -- прямая Симсона точки $B$ относительно треугольника $DKL$.

            По \cref{th:simson's line} $BF \perp KL$.
        }

    \item Пусть $AA_1$, $BB_1$, $CC_1$ -- высоты остроугольного треугольника $ABC$. Докажите, что проекции точки $A_1$ на прямые $AB$, $AC$, $BB_1$, $CC_1$ коллинеарны. \wishlisted

        \solution{
            Заметим, что точка $A_1$ лежит на окружностях $(ABB_1)$ и $(ACC_1)$.

            Тогда по \cref{th:simson's line} проекции точки $A_1$ на прямые $AB, AC, BB_1$ коллинеарны. И проекции точки $A_1$ на прямые $AB, AC, CC_1$ коллинеарны. Т.к. проекции точки $A_1$ на прямые $AB, AC$ составляют прямую, и причем только одну, то все эти точки лежат на этой прямой.
        }

    \item \task{Обобщённая прямая Симсона}{$P$ -- произвольная точка описанной окружности треугольника $ABC$. Докажите, что точки $A_1$, $B_1$, $C_1$ на прямых $AC$, $BC$, $AB$ коллинеарны, когда выполняется равенство: $$\angle(AB, PC_1) = \angle(BC, PA_1) = \angle(AC, PB_1).$$}

    \item Точки $P$ и $Q$ лежат на описанной окружности треугольника $ABC$. На прямой $AB$ выбрана точка $C_1$ так, что $\angle(AB, PC_1) = \angle(QC_1, AB)$. Аналогично выбраны точки $B_1$ и $C_1$ на прямых $AC$ и $BC$ соответственно. Докажите, что точки $A_1$, $B_1$, $C_1$ коллинеарны.

    \item Вписанная в треугольник $ABC$ окружность касается сторон $AB$, $BC$, $CA$ в точках $C_1$, $B_1$, $A_1$ соответственно. Пусть прямая $C_1I$ пересекает прямую $A_1B_1$ в точке $P$. Тогда прямая $CP$ содержит медиану треугольника $ABC$. \wishlisted

    \item \begin{enumerate}
        \item Хорда $PQ$ описанной окружности треугольника $ABC$ и сторона $BC$ перпендикулярны. Докажите, что прямая Симсона точки $P$ относительно треугольника $ABC$ параллельна прямой $AQ$. \wishlisted

        \item \task{Закл. этап ВСОШ, 2009–2010 гг., 10.6}{Пусть $H$ -- ортоцентр треугольника $ABC$. Точки $X$ и $Y$ -- проекции точки $P$, лежащей на описанной окружности треугольника $ABC$ на стороны $AB$ и $BC$. Докажите, что середина отрезка $HP$ и точки $X$ и $Y$  коллинеарны.\footnote{Подсказка в том, что эта задача -- пункт (b). Ну и симметрии ортоцентра.}}

    \end{enumerate}
    
    \item \task{Прямая Штейнера}{Пусть $P$ -- произвольная точка на описанной окружности треугольника $ABC$. Точки $P_a$, $P_b$, $P_c$ -- симметричны $P$ относительно прямых $BC$, $AC$ и $AB$ соответственно. Докажите что, точки $P_a$, $P_b$, $P_c$, $H$ коллинеарны.} \wishlisted

    \item Пусть $\ell$ -- прямая Штейнера точки $R$ на описанной окружности $ABC$. Докажите, что если прямую $\ell$ отразить относительно стороны треугольника $ABC$, то полученная прямая пройдет через точку $R$.

    \item \task{Л. А. Попов, Ф. Л. Бахарев}{Точки $A_1$, $B_1$, $C_1$ — основания высот остроугольного треугольника $ABC$ из точек $A$, $B$, $C$ соответственно. Точки $A_1$, $B_1$, $C_1$ отразили относительно средних линий треугольника, параллельных $AB$, $BC$, $CA$ соответственно, — получились точки $A_2$, $B_2$, $C_2$ соответственно. Докажите, что прямые $AA_2$, $BB_2$, $CC_2$ пересекаются в одной точке.}

    \item \task{Теорема Дроз-Фарни}{Обозначим точкой $H$ -- ортоцентр треугольника $ABC$. Прямые $\ell$ и $t$ проходят через $H$ и $\ell \perp t$. Пусть $L_a$, $L_b$, $L_c$ пересечение $\ell$ с прямыми $BC$, $AC$ и $AB$ соответственно, точки $T_a$, $T_b$ и $T_c$ определяются аналогично. Докажите, что середины отрезков $T_aL_a$, $T_bL_b$, $T_cL_c$ коллинеарны.}
    
    \item \task{Олимпиада им. И.Ф. Шарыгина, 2021, 8-9.6, устный тур}{\\В треугольнике $ABC$, точка $M$ -- середина стороны $BC$, точка $H$ -- ортоцентр. Биссектриса угла $A$ пересекает отрезок $HM$ в точке $T$. Окружность построенная на отрезке $AT$, как на диаметре, пересекает стороны $AB$ и $AC$ в точках $X$ и $Y$. Докажите, что точки $X$, $Y$ и $H$  коллинеарны.}
\end{enumerate}
