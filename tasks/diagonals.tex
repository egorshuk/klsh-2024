\begin{enumerate}[resume*]
    \item Докажите, что высоты треугольника конкурентны. \texttt{;)} \wishlisted

    \solution{
        Пусть $H$ -- точка пересечения высот $BH_B$ и $CH_C$, тогда будем доказывать, что $AH \perp BC$.

        В четырехугольнике $ABCH$ $BH\perp AC$, значит
        \begin{equation}\label{eq:26.1}
            AB^2 + CH^2 = BC^2 + AH^2.
        \end{equation}
        Также в этом четырехугольнике $CH \perp AB$, значит 
        
        \begin{equation}\label{eq:26.2}
            AC^2 + BH^2 = BC^2 + AH^2.
        \end{equation}

        По \cref{eq:26.1,eq:26.2} 
        \begin{equation}
            AC^2 + BH^2 = AB^2 + CH^2 \Leftrightarrow AH \perp BC.
        \end{equation}
    }

    \item \task{Муниципальный этап ВСОШ (Москва), 2020, 9.4}{Пусть точки $B$ и $C$ лежат на по\-лу\-окруж\-но\-сти с диаметром $AD$. Точка $M$ -- середина отрезка $BC$. Точка $N$ такова, что точка $M$ -- середина отрезка $AN$, докажите что $BC \perp ND$}. 

    \solution{
        \begin{theorem}[Пифагора]\label{th:pythagorean}
            В прямоугольном треугольник $ABC$ ($C$ -- прямой) \[AC^2 + BC^2 = AB^2\] 
        \end{theorem}

        $ABNC$ -- параллелограмм. Тогда раз AD -- диаметр, то \(AB \perp BD \) и \(AC \perp CD \). Применим теорему \labelcref{th:pythagorean} для треугольников $ABD$ и $ACD$.

        \begin{equation}\label{eq:28.1}
            AB^2 + BD^2 = AD^2.
        \end{equation}

        \begin{equation}\label{eq:28.2}
            AC^2 + CD^2 = AD^2.
        \end{equation}

        Тогда по \cref{eq:28.1,eq:28.2}

        \begin{equation}\label{eq:28.3}
            AC^2 + CD^2 = AB^2 + BD^2.
        \end{equation}

        По \cref{th:diagonals} \(BC \perp ND \Leftrightarrow BN^2 + CD^2 = BD^2 + NC^2\). 
        Т.к. $ABNC$ -- параллелограмм, то \(AB = NC\) и \(AC = BN\). Подставляем в \cref{eq:28.3}, получаем то, что нужно.
    }
    
    \item \task{Baltic Way, 2019, problem 13}{Let $ABCDEF$ be a convex hexagon in which $AB = AF$, $BC = CD$, $DE = EF$ and $\angle ABC = \angle EFA = 90^\circ$. Prove that $AD \perp CE$}. \wishlisted

    \item \begin{enumerate}
            \item \task{Теорема Штейнера}\label{th:shtainer}{Пусть $ABC$ и $A_1B_1C_1$ — невырожденные треугольники. Докажите, что перпендикуляры, опущенные из точек $A_1$, $B_1$, $C_1$ на прямые $BC$, $AC$, $AB$ пересекаются в одной точке тогда и только тогда, когда $$C_1A^2 + A_1B^2 + B_1C^2 = C_1B^2 + B_1A^2 + A_1C^2.$$} 

        \solution{
            Предположем, что перпендикуляры пересеклись в одной точке $M$. Тогда для каждого четырехугольника $AC_1BM$, $BA_1CM$, $CB_1AM$ выполняется теорема \labelcref{th:diagonals}.
            \begin{equation}
                    \begin{cases}
                        C_1A^2 + BM^2 &= C_1B^2 + AM^2 \\
                        A_1B^2 + CM^2 &= A_1C^2 + BM^2 \\
                        B_1C^2 + AM^2 &= B_1A^2 + CM^2
                    \end{cases}
            \end{equation}

            Вычитанием одинаковых слагаемых нужного следующего уравнения.
            \begin{equation}
                C_1A^2 + A_1B^2 + B_1C^2 = C_1B^2 + B_1A^2 + A_1C^2.
            \end{equation}
        }

    \item\label{th:orthologyoftriagnles} Докажите, что если перпендикуляры, опущенные из точек $A_1$, $B_1$, $C_1$ на прямые $BC$, $AC$, $AB$ пересекаются в одной точке, то и перпендикуляры, опущенные из точек $A$, $B$, $C$ на прямые $B_1C_1$, $A_1C_1$, $A_1B_1$ тоже.\footnote{Треугольники $ABC$ и $A_1B_1C_1$ из задачи называют ортологичными. Пишут $\triangle ABC \perp \triangle A_1B_1C_1$. При этом точки пересечения соответствующих перпендикуляров называют центрами ортологии.} \wishlisted

        \solution{
            Заметим, что при преобразовании \(S: x \leftrightarrow x_1\) уравнение из теоремы \labelcref{th:shtainer} переходит в себя.
        }
    \end{enumerate}

    \item \task{Теорема об изогональном сопряжении}{Чевианы $AA_1$, $BB_1$, $CC_1$ треугольника $ABC$ пересекаются в одной точке. Докажите, что чевианы, симметричные им относительно биссектрис соответствующих углов, тоже пересекаются в одной точке. \setcounter{footnote}{1}\footnote{Рассмотрите педальный треугольник этой точки.}} \wishlisted

        \solution{
            \begin{definition}[Педальный треугольник]\label{def:pedal}
                Педальный (подерный) треугольник точки $P$ относительно треугольника $ABC$ -- это треугольник, вершинами которого являются проекции точки $P$ на прямые $AB$, $AC$ и $BC$.
            \end{definition}
            Пусть \(AA_1 \cap BB_1 \cap CC_1 = P\). Построим $P_aP_bP_c$ -- педальный треугольник точки $P$. Тогда \(P_aP_bP_c \perp ABC\), а $P$ -- является одним из центров ортологии треугольников $ABC$ и $P_aP_bP_c$. 
            Построим второй центр ортологии $Q$. Для этого проведем чевианы $AA_2 \perp P_bP_c$, $BB_2 \perp P_aP_c$, $CC_2 \perp P_aP_b$. Эти чевианы по \cref{th:orthologyoftriagnles} пересекаются в точке $Q$.

            Рассмотрим треугольник $AP_bP_c$. В нем $AQ$ содержит в себе высоту. Точка $P$ лежит на окружности $(AP_cP_b)$, т.к. \(\angle AP_cP = \angle AP_bP = 90^\circ\), значит точка $P$ диаметрально противоположна точке $A$. По \cref{th:OHisogonal} \(\angle PAB = \angle QAC\). Аналогично для треугольников $BP_cP_a$ и $CP_aP_b$. 
        }

    \item Пусть точки $P$ и $Q$ -- изогонально сопряженные точки треугольника $ABC$. $B_p$, $C_p$ и $B_q$, $C_q$ -- перпендикуляры из $P$ и $Q$ на прямые $AC$ и $AB$ соответственно. \begin{enumerate}
        \item Докажите, что треугольники $PB_pC_p$ и $QB_qC_q$ подобны.
        \item Докажите, что вершины педальных треугольников изогонально сопряженных точек лежат на одной окружности. Найдите её центр. \wishlisted
    \end{enumerate}
    
    \item Углы $A$ и $C$ четырехугольника $ABCD$ равны. Докажите, что середина отрезка $AC$ и проекции $D$ на прямые $AB$, $BC$ и $AC$ концикличны.
\end{enumerate}
