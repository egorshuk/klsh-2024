\begin{enumerate}[resume*]
    \item Докажите, что высоты треугольника конкурентны. \texttt{;)} \wishlisted

    \solution{
        Пусть $H$ -- точка пересечения высот $BH_B$ и $CH_C$, тогда будем доказывать, что $AH \perp BC$.

        В четырехугольнике $ABCH$ $BH\perp AC$, значит
        \begin{equation}\label{eq:26.1}
            AB^2 + CH^2 = BC^2 + AH^2.
        \end{equation} Также в этом четырехугольнике $CH \perp AB$, значит 
        
        \begin{equation}\label{eq:26.2}
            AC^2 + BH^2 = BC^2 + AH^2.
        \end{equation}

        По \cref{eq:26.1,eq:26.2} 
        \begin{equation}
            AC^2 + BH^2 = AB^2 + CH^2 \Leftrightarrow AH \perp BC.
        \end{equation}
    }

    \item \task{Муниципальный этап ВСОШ (Москва), 2020, 9.4}{Пусть точки $B$ и $C$ лежат на по\-лу\-окруж\-но\-сти с диаметром $AD$. Точка $M$ -- середина отрезка $BC$. Точка $N$ такова, что точка $M$ -- середина отрезка $AN$, докажите что $BC \perp ND$}. 
    
    \item \task{Baltic Way, 2019, problem 13}{Let $ABCDEF$ be a convex hexagon in which $AB = AF$, $BC = CD$, $DE = EF$ and $\angle ABC = \angle EFA = 90^\circ$. Prove that $AD \perp CE$}. \wishlisted

    \item \begin{enumerate}
        \item \task{Теорема Штейнера}{Пусть $ABC$ и $A_1B_1C_1$ — невырожденные треугольники. Докажите, что перпендикуляры, опущенные из точек $A_1$, $B_1$, $C_1$ на прямые $BC$, $AC$, $AB$ пересекаются в одной точке тогда и только тогда, когда $$C_1A^2 + A_1B^2 + B_1C^2 = C_1B^2 + B_1A^2 + A_1C^2.$$} 
        \item Докажите, что если перпендикуляры, опущенные из точек $A_1$, $B_1$, $C_1$ на прямые $BC$, $AC$, $AB$ пересекаются в одной точке, то и перпендикуляры, опущенные из точек $A$, $B$, $C$ на прямые $B_1C_1$, $A_1C_1$, $A_1B_1$ тоже.\footnote{Треугольники $ABC$ и $A_1B_1C_1$ из задачи называют ортологичными. Пишут $\triangle ABC \perp \triangle A_1B_1C_1$. При этом точки пересечения соответствующих перпендикуляров называют центрами ортологии.} \wishlisted
    \end{enumerate}

    \item \task{Теорема об изогональном сопряжении}{Чевианы $AA_1$, $BB_1$, $CC_1$ треугольника $ABC$ пересекаются в одной точке. Докажите, что чевианы, симметричные им относительно биссектрис соответствующих углов, тоже пересекаются в одной точке. \setcounter{footnote}{1}\footnote{Рассмотрите педальный треугольник этой точки.}} \wishlisted

    \item Пусть точки $P$ и $Q$ -- изогонально сопряженные точки треугольника $ABC$. $B_1$, $C_1$ и $B_2$, $C_2$ -- перпендикуляры из $P$ и $Q$ на прямые $AC$ и $AB$ соответственно. \begin{enumerate}
        \item Докажите, что треугольники $P_1B_1C_1$ и $P_2B_2C_2$ подобны.
        \item Докажите, что вершины педальных треугольников изогонально сопряженных точек лежат на одной окружности. Найдите её центр. \wishlisted
    \end{enumerate}
    
    \item Углы $A$ и $C$ четырехугольника $ABCD$ равны. Докажите, что середина отрезка $AC$ и проекции $D$ на прямые $AB$, $BC$ и $AC$ концикличны.
\end{enumerate}