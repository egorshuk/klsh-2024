\documentclass[12pt]{article}
\usepackage[utf8]{inputenc}
\usepackage[T2A]{fontenc}
\usepackage{amsmath, amsfonts, amssymb, arcs}
\usepackage{geometry, graphicx, hyperref, enumitem, indentfirst, authblk}
\usepackage[english, russian]{babel}

\geometry{
    a4paper,
    right = 1cm,
    left = 1cm,
    top = 2cm,
    bottom = 2 cm,
    footnotesep = 5mm,
}
\setlength{\headheight}{16pt}
\interfootnotelinepenalty=10000

\renewcommand{\labelenumi}{(\alph{enumi})}

\hypersetup{
    colorlinks=true,
    linkcolor=red,
    urlcolor=blue,
    pdftitle={Зондерская задача},
    pdfauthor={Лунёв Егор}
}


\usepackage[inline]{asymptote}
\begin{asydef}
    settings.outformat = "pdf";
    import geometry;
\end{asydef}

\date{До 1 июля}
\title{Зондерская задача по планиметрии}
\author[1]{Егор Лунёв}
\affil[1]{Зондер комнады $\tau$, cеминарист у Тони Шарковской на \texttt{У1}, семинарист у Яна Шапиро на \texttt{У2}}

\begin{document}

    \clearpage\maketitle
    \thispagestyle{empty}
    
    В треугольнике $ABC$, точка $M$ -- середина стороны $BC$, точка $H$ -- ортоцентр\footnote{Точка пересечения высот.}. Биссектриса угла $A$ пересекает отрезок $HM$ в точке $T$. Окружность построенная на отрезке $AT$, как на диаметре, пересекает стороны $AB$ и $AC$ в точках $P$ и $Q$. Докажите, что точки $P$, $Q$ и $H$ коллинеарны\footnote{Лежат на одной прямой.}.

    \vspace{5mm}
        
    \emph{Подсказка: воспользуйтесь фактом о прямой Симсона-Уолесса.}

    \vspace{1cm}

    \begin{figure}[h]
        \centering
        \begin{asy}
            size(15cm, 13cm);
            triangle t = triangleabc(9.4, 6.7, 9.1);
            draw(t, linewidth(bp)); 

            point M = midpoint(t.AB);
            draw(t.A--t.B, StickIntervalMarker(2, 1, size=10));
            point H = orthocentercenter(t); 

            point Ha = foot(t.BC);
            point Hb = foot(t.VB);
            point Hc = foot(t.VC);
            draw(t.A--Ha, gray);
            draw(t.B--Hb, gray);
            draw(t.C--Hc, gray);
            perpendicularmark(t.AB, altitude(t.AB), size=12);
            perpendicularmark(reverse(t.AC), altitude(t.AC), size=12);
            perpendicularmark(t.BC, altitude(t.BC), size=12);

            draw(H--M);

            line b = bisector(t.VC); 
            point T = intersectionpoint(b, line(H,M)); 
            draw(t.C--T);

            markangle(n = 1, t.A, t.C, T, radius = 30);
            markangle(n = 1, T, t.C, t.B, radius = 40);

            point G = midpoint(segment(t.VC, T));

            circle omega = circle(G, length(segment(t.VC, G)));
            draw(omega);
            
            point P = intersectionpoints(omega, t.AC)[0];
            point Q = intersectionpoints(omega, t.BC)[0];


            draw(line(P,Q), dashed);

            dot("$B$", t.A, dir(200), black+1);
            dot("$A$", t.C, dir(140), black+1);
            dot("$C$", t.B, dir(40), black+1);
            dot(G, filltype=FillDraw(fillpen=white, drawpen=black), black+1);            
            dot("$P$", P, dir(-135), black+5);
            dot("$Q$", Q, dir(50), black+5);
            dot("$T$", T, dir(-100), black+5);
            dot("$H$", H, dir(120), black+5);
            dot("$M$", M, dir(-80), black+5);
        \end{asy}
    \end{figure}

\end{document}