\documentclass[12pt, twoside]{article}
\usepackage[utf8]{inputenc}
\usepackage[T2A]{fontenc}
\usepackage{amsmath, amsfonts, amssymb, arcs}
\usepackage{geometry, graphicx, hyperref, fancyhdr, enumitem, indentfirst, multicol}
\usepackage[english, russian]{babel}

\geometry{
	a4paper,
	right = 1cm,
	left = 1cm,
	top = 2cm,
	bottom = 2 cm,
	footnotesep = 5mm,
}
\setlength{\headheight}{16pt}
\interfootnotelinepenalty=10000

\hypersetup{
	colorlinks=true,
	linkcolor=red,
	urlcolor=blue,
	pdftitle={Топ-5 теорем, которые не помогут построить дом},
	pdfauthor={Лунёв Егор}
}

\pagestyle{fancy}
\fancyhead{} 
\fancyhead[L]{КЛШ-2024}
\fancyhead[R]{\href{https://t.me/standing_on_my_neck}{\color{red}{$\psi_c$}} $\perp$ \href{https://t.me/egorrshuk}{\color{blue}{$\omega_t$}}}

\fancyfoot{}
\fancyfoot[C]{\thepage}
\setlength{\tabcolsep}{12pt}
\renewcommand{\arraystretch}{1.4}
\setlength{\arrayrulewidth}{0.5mm}


\setlength{\columnsep}{1cm}
\setlength{\columnseprule}{1pt}
\def\columnseprulecolor{\color{blue}}

\newcommand{\task}[2]{\texttt{(#1)} #2}

\let\ch\undefined
\renewcommand{\labelenumii}{(\alph{enumii})}
\counterwithin*{footnote}{page}
\newcommand{\frule}[2]{\noindent \textcolor{red}{\rule{#1\textwidth}{0.5pt}}\textcolor{blue}{\rule{#2\textwidth}{0.5pt}}}
\renewcommand{\footnoterule}{\vspace{1mm}\frule{0.247}{0.153}\vspace{1mm}}
\newcounter{wishlist}
\newcommand{\wishlisted}{\hfill{$\bigcirc$}\stepcounter{wishlist}}

\newlist{tasks}{itemize}{2}

\setlist[tasks, 1]{
	label=\color{red}$\heartsuit$
}

\usepackage[inline]{asymptote}
\begin{asydef}
	settings.outformat = "pdf";
	import geometry;
\end{asydef}

\begin{document}
	
	\begin{center}
		
		\Large
		\textbf{Подборка красивых задач для ВИПа}
		\large
		\vspace{5mm}
		
		Топ-5 теорем, которые не помогут построить дом
		
		\vspace{5mm}
		
		Лунёв Егор (\href{https://t.me/egorrshuk}{\texttt{@egorrshuk}})
		
		\vspace{5mm}
	\end{center}
	
	\begin{multicols}{2}
		\begin{tasks}
			\item \task{Теорема об изогональном сопряжении}{Чевианы $AA_1$, $BB_1$, $CC_1$ треугольника $ABC$ пересекаются в одной точке. Докажите, что чевианы, симметричные им относительно биссектрис соответствующих углов, тоже пересекаются в одной точке. \footnote{Рассмотрите педальный треугольник этой точки.}}
			
			\begin{center}
				\begin{asy}
				size(8cm);
				
				triangle t = triangleabc(18.5, 15, 19.5);
				draw(t, linewidth(1.4*bp));
				point P = 0.47*t.B+0.3*(t.C-t.B); 
				point Q = isogonalconjugate(t, P);
				
				segment pa = segment(t.VA, P); draw(pa, blue);
				segment pb = segment(t.VB, P); draw(pb, blue); 
				segment pc = segment(t.VC, P); draw(pc, blue);
				
				segment qa = segment(t.VA, Q); draw(qa, red+dashed);
				segment qb = segment(t.VB, Q); draw(qb, red+dashed); 
				segment qc = segment(t.VC, Q); draw(qc, red+dashed);
				
				markangle(n = 1, line(pa), line(t.AC), radius=1.3cm, blue);
				markangle(n = 1, line(t.AB), line(qa), radius=1cm, red);
				
				markangle(n = 2, rotate(180)*line(t.AC), line(pc), radius=1.3cm, blue);
				markangle(n = 2, line(qc), line(t.BC), radius=1cm, red);
				
				markangle(n = 3, line(t.BC), line(qb), radius=1.3cm, red);
				markangle(n = 3, line(pb), line(t.AB), radius=1.4cm, blue);
				
				dot(P, red+4);
				dot(Q, blue+4);
				\end{asy}
			\end{center}
			
			\item Окружности ${\color{red}{\psi}}$ и ${\color{blue}{\omega}}$ вписаны в вертикальный угол $\angle {\color{red}{n}}{\color{blue}{m}}$, ${\color{red}{\psi}}$ касается прямой ${\color{red}{n}}$ в точке ${\color{red}{N}}$, а ${\color{blue}{\omega}}$ касается прямой ${\color{blue}{m}}$ в точке ${\color{blue}{M}}$. Докажите, что ${\color{red}{\psi}}$ и ${\color{blue}{\omega}}$ высекают на ${\color{red}{N}}{\color{blue}{M}}$ равные отрезки. 
			
			\begin{center}
				\begin{asy}
				size(8cm);
				
				circle psi = circle(origin(), 1); draw(psi, red);
				
				point O = (-2, -1); 
				
				line[] tangents = tangents(psi, O);
				draw(tangents[0], red);
				point N = intersectionpoints(tangents[0], psi)[0];
				
				
				point sdf = angpoint(psi, 90); 
				
				point Ns = scale(0.6, O)*N; 
				point sdfs = scale(0.6, O)*sdf;
				
				circle omega = circle(Ns, sdfs); draw(omega, blue);
				
				draw(tangents[1], blue);
				
				point M = intersectionpoints(tangents[1], omega)[0];
				
				line mn = line(N, M);
				
				point P = intersectionpoints(psi, mn)[1]; 
				point Q = intersectionpoints(omega, mn)[0]; 
				
				draw(M--P, dashed,StickIntervalMarker(1, 1, size=7));
				draw(N--Q, dashed,StickIntervalMarker(1, 1, size=7));
				draw(P--Q, dashed);
				
				
				
				dot(P);
				dot(Q);
				dot(M, blue);
				dot(N, red);
				
				
				draw(box((-1.6, 0),(0, 1)), invisible);
				\end{asy}
			\end{center}
			
			\item \task{Теорема Гаусса-Боденмиллера}{Докажите, что прямая Гаусса \footnote{Прямой Гаусса полного четырёхсторонника называется прямая, проходящая через середины трех его диагоналей.} перпендикулярна оси Обера.}
			
			\item \task{Прямая Штейнера}{Пусть $P$ -- произвольная точка на описанной окружности треугольника $ABC$. Точки $P_a$, $P_b$, $P_c$ -- симметричны $P$ относительно прямых $BC$, $AC$ и $AB$ соответственно. Докажите что, точки $P_a$, $P_b$, $P_c$, $H$ коллинеарны.}
			
			\begin{center}
				\begin{asy}
				size(8cm); 
				triangle t = triangleabc(11.1, 12.3, 11); draw(t, linewidth(bp));
				
				circle omega = circle(t.A, t.B, t.C); draw(omega);
				
				point H = orthocentercenter(t); 
				
				point Ha = foot(t.BC);
				point Hb = foot(t.VB);
				point Hc = foot(t.VC);
				draw(t.A--Ha, gray);
				draw(t.B--Hb, gray);
				draw(t.C--Hc, gray);
				perpendicularmark(t.AB, altitude(t.AB), size=7;
				perpendicularmark(reverse(t.AC), altitude(t.AC), size=7);
				perpendicularmark(t.BC, altitude(t.BC), size=7);
				
				point P = angpoint(omega, -70); 
				point Pa = reflect(t.BC)*P;
				point Pb = reflect(t.AC)*P; 
				point Pc = reflect(t.AB)*P; 
				
				draw(P--Pa, dashed+blue, StickIntervalMarker(2, 2, size=7));
				draw(P--Pb, dashed+blue, StickIntervalMarker(2, 3, size=7));
				draw(P--Pc, dashed+blue, StickIntervalMarker(2, 1, size=7));
				
				
				
				
				dot(Pa);
				dot(Pb);
				dot(Pc);
				dot(P, black+4);
				dot(H, black+4);
				\end{asy}
			\end{center}
			
			\item \task{Олимпиада им. И.Ф. Шарыгина, 2021, 8-9.6, устный тур}{\\В треугольнике $ABC$, точка $M$ -- середина стороны $BC$, точка $H$ -- ортоцентр. Биссектриса угла $A$ пересекает отрезок $HM$ в точке $T$. Окружность построенная на отрезке $AT$, как на диаметре, пересекает стороны $AB$ и $AC$ в точках $X$ и $Y$. Докажите, что точки $X$, $Y$ и $H$  коллинеарны.}
			
			\begin{center}
				\begin{asy}
				size(8cm);
				triangle t = triangleabc(9.4, 6.7, 9.1);
				draw(t, linewidth(bp)); 
				
				point M = midpoint(t.AB);
				draw(t.A--t.B, StickIntervalMarker(2, 1, size=10));
				point H = orthocentercenter(t); 
				
				point Ha = foot(t.BC);
				point Hb = foot(t.VB);
				point Hc = foot(t.VC);
				draw(t.A--Ha, gray);
				draw(t.B--Hb, gray);
				draw(t.C--Hc, gray);
				perpendicularmark(t.AB, altitude(t.AB), size=10);
				perpendicularmark(reverse(t.AC), altitude(t.AC), size=10);
				perpendicularmark(t.BC, altitude(t.BC), size=10);
				
				draw(H--M);
				
				line b = bisector(t.VC); 
				point T = intersectionpoint(b, line(H,M)); 
				draw(t.C--T);
				
				markangle(n = 1, t.A, t.C, T, radius = 30, blue);
				markangle(n = 1, T, t.C, t.B, radius = 25, blue);
				
				point G = midpoint(segment(t.VC, T));
				
				circle omega = circle(G, length(segment(t.VC, G)));
				draw(omega, red);
				
				point P = intersectionpoints(omega, t.AC)[0];
				point Q = intersectionpoints(omega, t.BC)[0];
				
				
				draw(line(P,Q), dashed+blue);
				dot(G, filltype=FillDraw(fillpen=white, drawpen=black), black+0.7);
				dot(P);
				dot(Q);
				dot(T);
				dot(H, black+4);
				dot(M);
				\end{asy}
			\end{center}
			
			\item \task{Теорема Дроз-Фарни}{Обозначим точкой $H$ -- ортоцентр треугольника $ABC$. Прямые $\ell$ и $t$ проходят через $H$ и $\ell \perp t$. Пусть $L_a$, $L_b$, $L_c$ пересечение $\ell$ с прямыми $BC$, $AC$ и $AB$ соответственно, точки $T_a$, $T_b$ и $T_c$ определяются аналогично. Докажите, что середины отрезков $T_aL_a$, $T_bL_b$, $T_cL_c$ коллинеарны.}
			
			\begin{center}
				\begin{asy}
				size(8cm);
				triangle tri = triangleabc(8, 7, 8.5);
				point H = orthocentercenter(tri); 
				
				point Ha = foot(tri.BC);
				point Hb = foot(tri.VB);
				point Hc = foot(tri.VC);
				draw(tri.A--Ha, gray);
				draw(tri.B--Hb, gray);
				draw(tri.C--Hc, gray);
				perpendicularmark(tri.AB, altitude(tri.AB), size=7);
				perpendicularmark(reverse(tri.AC), altitude(tri.AC), size=7);
				perpendicularmark(tri.BC, altitude(tri.BC), size=7);
				
				line t = line(H, 27); draw(t, red); line l = line(H, 117); draw(l, blue);
				
				point T_a = intersectionpoint(t, tri.BC);
				point T_b = intersectionpoint(t, tri.AC); 
				point T_c = intersectionpoint(t, tri.BA); 
				
				point L_a = intersectionpoint(l, tri.BC); 
				point L_b = intersectionpoint(l, tri.AC); 
				point L_c = intersectionpoint(l, tri.BA); 
				
				draw(line(tri.VA, T_c));
				draw(line(tri.VC, L_a));
				draw(segment(tri.VA, tri.VC));
				
				point M_a = midpoint(segment(L_a, T_a));
				point M_b = midpoint(segment(L_b, T_b));
				point M_c = midpoint(segment(L_c, T_c));
				
				draw(segment(L_a, T_a), StickIntervalMarker(2, 2, size=8));
				draw(segment(L_b, T_b), StickIntervalMarker(2, 1, size=6));
				draw(segment(L_c, T_c),StickIntervalMarker(2, 3, size=8));
				
				perpendicularmark(l, t, size=8, quarter=3);
				
				draw(line(M_a, M_c), dashed);
				
				dot(T_a); dot(T_b); dot(T_c);
				dot(L_a); dot(L_b); dot(L_c);
				dot(M_a, black+4); dot(M_b, black+4); dot(M_c, black+4);
				dot(H, 4+black);
				
				draw(box((-3.7, -1), (9.5, 12.5)), invisible);
				\end{asy}
			\end{center}
		\end{tasks}
	\end{multicols}
\end{document}