Ортоцентр -- это такая особенная точка: конструкции, в которых используются его \textbf{симметрии} относительно чего-либо, \textbf{замечательно} связанны с описанной окружностью, и наоборот!

\noindent
\begin{minipage}{0.55\textwidth}
    \begin{theorem}\label{th:side reflect}
        Если отразить ортоцентр относительно стороны, то он попадет на описанную окружность.
    \end{theorem}
\end{minipage}
\hspace{0.05\textwidth}
\begin{minipage}{0.4\textwidth}
    \begin{asy}
        size(5cm, 4cm);
        triangle t = triangleabc(11.1, 12.3, 11); draw(t, linewidth(bp));
        circle omega = circle(t); draw(omega);
        point H = orthocentercenter(t);

        point B_1 = intersectionpoints(altitude(t.VB), omega)[1]; 
        point A_1 = intersectionpoints(altitude(t.VA), omega)[1];

        point B_2 = foot(t.VB); point A_2 = foot(t.VA);

        draw(segment(H, B_1), dashed+red, StickIntervalMarker(2, 2, size=5));
        draw(segment(H, A_1), dashed+red, StickIntervalMarker(2, 1, size=5));

        perpendicularmark(t.AC, line(H, B_1), size=8, quarter=3, blue);
        perpendicularmark(t.BC, line(H, A_1), size=7, quarter=2, blue);
        
        dot(B_2); dot(A_2); dot(B_1); dot(A_1);
        dot(H, hpen);

        //draw(box((-2, 0), (14, 10.5)), invisible);
    \end{asy}
\end{minipage}\vspace{0.03\textwidth}
\begin{minipage}{0.55\textwidth}
    \begin{theorem}\label{th:middle reflect}
        Если ортоцентр отразить относительно середины стороны, то он попадет на описанную окружность.
    \end{theorem}
    \label{th 2.2}
\end{minipage}
\hspace{0.05\textwidth}
\begin{minipage}{0.4\textwidth}
    \begin{asy}
        size(5cm, 4cm);
        triangle t = triangleabc(9.6, 10.8, 12); draw(t, linewidth(bp));
        point H = orthocentercenter(t);
        
        circle Omega = circle(t); draw(Omega);
        point M_a = midpoint(t.CB);
    
        draw(segment(t.VC, t.VB), linewidth(bp), StickIntervalMarker(2, 1, size=7));
        
    
        point H_a = intersectionpoints(line(H, M_a), Omega)[0];

        dot(M_a); dot(H_a);
        draw(segment(H, H_a), red+dashed, StickIntervalMarker(2,2, size=7));
        dot(H, hpen);

        //draw(box((-1, -1), (13, 11)), invisible);
    \end{asy}
\end{minipage}\vspace{0.03\textwidth}
\begin{minipage}{0.55\textwidth}
    \begin{corollary}\label{cor:diametr}
        Точка из теоремы \ref{th:middle reflect} диаметрально противоположна противолежащей стороне вершине.
    \end{corollary}
\end{minipage}
\hspace{0.05\textwidth}
\begin{minipage}{0.4\textwidth}
    \begin{asy}
        size(5cm, 4cm);
        triangle t = triangleabc(11.8, 14, 11); draw(t, linewidth(bp)); 
        point H = orthocentercenter(t);
        point O = circumcenter(t);
        
        circle Omega = circle(t); draw(Omega);
        point M_a = midpoint(t.CB);
    
        draw(segment(t.VC, t.VB), linewidth(bp), StickIntervalMarker(2, 1, size=7));
        
        point H_a = intersectionpoints(line(H, M_a), Omega)[1];
        draw(segment(H, H_a), gray, StickIntervalMarker(2,2, size=7));
        draw(segment(t.VA, H_a), red);

        dot(H_a); dot(M_a);
        dot(H, hpen);
        dot(O, filltype=FillDraw(fillpen=white, drawpen=black));

        //draw(box((-2, -1), (13, 12)), invisible);
    \end{asy}
\end{minipage}\vspace{0.03\textwidth}
\begin{minipage}{0.55\textwidth}
    \begin{corollary}\label{cor:distance from O}
        Расстояние от вершины треугольника до ортоцентра в $2$ раза больше расстояния от центра описанной окружности до противолежащей стороны.         
    \end{corollary}
\end{minipage}
\hspace{0.05\textwidth}
\begin{minipage}{0.4\textwidth}
    \begin{asy}
        size(5cm, 4cm);
        triangle t = triangleabc(9.5, 12.5, 12.2); draw(t, linewidth(bp));
        circle omega = circle(t); 
        point H = orthocentercenter(t);
        point O = circumcenter(t);
        point M = midpoint(t.AB);

        draw(segment(t.VA, t.VB), StickIntervalMarker(2, 2, size=5));
        perpendicularmark(t.AB, line(O, M), blue, size=7);

        draw(segment(O, M), gray, StickIntervalMarker(1, 1, size=5)); 
        draw(segment(t.VC, H), red, StickIntervalMarker(2, 1, size=5));
        dot(midpoint(segment(t.VC, H)));

        
        dot(H, hpen);
        dot(O, filltype=FillDraw(fillpen=white, drawpen=black));
    \end{asy}
\end{minipage}\vspace{0.03\textwidth}
\begin{minipage}{0.55\textwidth}
    \begin{lemma}[Окружность Джонсона]\label{lem:(ABH)}
        $(ABC) = (ABH)$, т.е. окружности, описанные вокруг $\triangle ABC$ и $\triangle ABH$ равны.
    \end{lemma}
\end{minipage}
\hspace{0.05\textwidth}
\begin{minipage}{0.4\textwidth}
    \begin{asy}
        size(5cm, 4cm);
        triangle t = triangleabc(9, 11, 10); draw(t, linewidth(bp)); 
        circle abc = circle(t); draw(abc);
        point H = orthocentercenter(t);
    
        draw(circle(point(t.VC), point(t.VB), H), dashed+red);
        dot(H, hpen);
    \end{asy}
\end{minipage}\vspace{0.03\textwidth}
\begin{minipage}{0.55\textwidth}
    \begin{definition}[Изогональное сопряжение\setcounter{footnote}{0}\footnotemark]\label{def:isogonal}
        Точки $P$, $Q$ называются изогонально сопряженными, если $\angle PAB = \angle QAC$, $\angle PBC = \angle QBA$, $\angle PCB = \angle QCA$.
    \end{definition}
    \begin{theorem}\label{th:OHisogonal}
        Ортоцентр и центр описанной окружности изогонально сопряжены.
    \end{theorem}
\end{minipage}
\footnotetext{Можно думать об изогональном сопряжении, как о симметрии относительно биссектрисы.}
\hspace{0.05\textwidth}
\begin{minipage}{0.4\textwidth}
    \begin{asy}
        size(5cm, 4cm);
        triangle t = triangleabc(11, 9, 12); draw(t, linewidth(bp));
        point H = orthocentercenter(t);
        point O = circumcenter(t);

        draw(segment(t.VA, H));
        draw(segment(t.VA, O));

        draw(segment(t.VC, H));
        draw(segment(t.VC, O));

        markangle(n = 1, line(t.AB), line(t.VA, O), radius=35, red);
        markangle(n = 1, line(t.VA, H), line(t.AC), radius=30, red+dashed);

        markangle(n = 2, line(t.VC, O), line(t.BC), radius=30, dashed+blue);
        markangle(n = 2, rotate(180)*line(t.AC), line(t.VC, H), radius=30, blue);

        dot(H, hpen);
        dot(O, filltype=FillDraw(fillpen=white, drawpen=black));
    \end{asy}
\end{minipage}
