Давайте соединим пару свойств, которые мы уже знаем (а именно по \cref{th:side reflect,th:middle reflect}) и сделаем парочку незамысловатых размышлений. Получим \emph{окружность Эйлера} или \emph{окружность девяти точек}.
\begin{definition}[Окружность Эйлера]\label{def:euler's circle}
    Окружностью Эйлера называют окружность, проходящую через основания высот, середины сторон и середины отрезков, соединяющих вершины с ортоцентром треугольника.
\end{definition}
\begin{definition}[Прямая Эйлера]\label{def:euler's line}
    Точки $O, O_9, H, M$ лежат на одной прямой, называемой прямой Эйлера.
\end{definition}
\begin{theorem} \label{th:euler's line ratios}
    Отрезки на прямой Эйлера хорошо относятся.\[\overrightarrow{O_9M} : \overrightarrow{MO}: \overrightarrow{OH} = 1 : 2 : (-3)\]
\end{theorem}

\begin{figure}[ht]
    \centering
    \begin{asy}
        size(12.7cm, 10cm);
        triangle t = triangleabc(14.5, 18, 19.5); draw(t, linewidth(bp));
        circle omega = circle(t);
        
        point H = orthocentercenter(t);

        point H_a = foot(t.VA); point H_b = foot(t.VB); point H_c = foot(t.VC);
        point M_a = midpoint(t.BC); point M_b = midpoint(t.AC); point M_c = midpoint(t.AB);
        point T_a = midpoint(segment(H, t.VA)); point T_b = midpoint(segment(H, t.VB)); point T_c = midpoint(segment(H, t.VC));

        triangle t_1 = triangle(H_a, H_b, H_c);

        draw(segment(t.VA, H_a), grey); draw(segment(t.VB, H_b), grey); draw(segment(t.VC, H_c), grey);
        perpendicularmark(t.AB, altitude(t.VC), blue, size=10); perpendicularmark(t.AC, altitude(t.VB), blue, size=10, quarter=3); perpendicularmark(t.CB, altitude(t.VA), blue, size=10, quarter=4);

        draw(segment(t.VA, t.VB), StickIntervalMarker(2, 1, size=8)); draw(segment(t.VA, t.VC), StickIntervalMarker(2, 2, size=8)); draw(segment(t.VC, t.VB), StickIntervalMarker(2, 3, size=8));

        draw(segment(t.VA, H), grey, StickIntervalMarker(2, 3, blue, size=5)); draw(segment(H, t.VC), grey, StickIntervalMarker(2, 1, blue, size=5)); draw(segment(H, t.VB), grey, StickIntervalMarker(2, 2, blue, size=5));
        
        draw(Label("$\omega_9$", Relative(0.375)), circle(t_1), red+dashed+0.8*bp);
        draw(H_a--H_b--H_c--H_a); draw(M_a--M_b--M_c--M_a);

        dot(H_a); dot(H_b); dot(H_c);
        dot(M_a); dot(M_b); dot(M_c);
        dot(T_a); dot(T_b); dot(T_c);
        
        
        draw(segment(t.VA, M_a), grey); draw(segment(t.VB, M_b), grey); draw(segment(t.VC, M_c), grey);

        draw(line(circumcenter(t), H), blue+0.8*bp+dashed);
        
        dot(centroid(t));
        dot("$O_9$", circumcenter(t_1), dir(0), red);
        dot(circumcenter(t), filltype=FillDraw(fillpen=white, drawpen=black));
        dot(H, hpen);
        
        draw(box((-1,-0.7), (10.5, 6.3)), invisible);
    \end{asy}
    \caption{Окружность Эйлера и прямая Эйлера.}
\end{figure}
