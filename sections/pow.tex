\begin{definition}[Степень точки]\label{def:pow}
    Степень точки $P$, находящейся на расстоянии $d$ от центра окружности $\omega$ радиусом $r$, относительно этой же окружности: $$\pow(P, \omega) = d^2-r^2.$$
\end{definition}

\begin{theorem}\label{th:tan}
    Если прямая $\ell \ni P$ касается окружность в точке $K$, то $$\pow(P, \omega) = PK^2.$$
\end{theorem}

\begin{theorem}\label{th:pow}
    Если прямая $\ell \ni P$ пересекает окружность $\omega$ в точках $A$ и $B$, тогда $$\pow(P, \omega) = \overrightarrow{PA} \cdot \overrightarrow{PB}.$$
\end{theorem}

\begin{figure}[ht]
    \centering
    \begin{asy}
        size(12cm, 6cm);
        point O = (2, 0);
        circle omega = circle(O, 1.7); draw(omega);
        point P = (-2.5, 2); dot("$P$", P, dir(70));

        point A_2 = angpoint(omega, 40); dot("$A_2$", A_2, dir(40));
        point A_1 = intersectionpoints(omega, line(P, A_2))[1]; dot("$A_1$", A_1, dir(100));
        draw(line(P, A_2));

        point B_2 = angpoint(omega, -80); dot("$B_2$", B_2, dir(-100));
        point B_1 = intersectionpoints(omega, line(P, B_2))[1]; dot("$B_1$", B_1, dir(-140));
        draw(line(P, B_2));

        draw(B_1--A_2, red+dashed); draw(B_2--A_1, red+dashed);

        markangle(n = 1, P, A_2, B_1, radius=25, blue);
        markangle(n = 1, A_1, B_2, P, radius=25, blue);

        markangle(n = 2, B_2, P, A_2, radius=20, blue+bp);

        draw(box((-3,-2), (4, 2.3)), invisible);
    \end{asy}
\end{figure}

\begin{corollary}[Теорема о касательной и секущей]\label{cor:tangent_and_sector}
    Если из точки $P$, проведена касательная $PK$ к окружности $\omega$ и прямая $(\ell \ni P)$ пересекает окружность $\omega$ в точках $A$ и $B$, тогда $$PK^2 = PA \cdot PB.$$
\end{corollary}

\begin{theorem}[Главная теорема о степени точки]\label{th:superpow}
    Если через точку $P$ проходят две прямые, которые пересекают окружность $\omega$ в точках $A_1, A_2$ и $B_1, B_2$ соответственно, то $$\pow(P, \omega) = \overrightarrow{PA_1}\cdot\overrightarrow{PA_2} = \overrightarrow{PB_1}\cdot\overrightarrow{PB_2}.$$
\end{theorem}
