\noindent Это самое базовое, что можно сделать, чтобы доказать перпендикулярность: просто посчитать углы, и из этого сделать вывод (какой-то угол будет равен $90^\circ$).
\begin{theorem}\label{th:otrhocenter}
    Высоты треугольника \emph{конкурентны\footnote{Пересекаются в одной точке.}}.
\end{theorem}
\begin{figure}[ht]
    \centering
    \begin{asy}
        size(9cm);
        triangle t = triangleabc(9, 10, 11); draw(t, linewidth(bp));

        point H_a = foot(t.VA); point H_b = foot(t.VB); point H_c = foot(t.VC);
        point H = orthocentercenter(t);

        draw(segment(t.VA, H_a), grey); draw(segment(t.VB, H_b), grey); draw(segment(t.VC, H_c), grey);

        perpendicularmark(t.AB, line(t.VC, H_c), dashed + blue, size=10);
        perpendicularmark(t.BC, line(t.VA, H_a), blue, size=10, quarter=1);
        perpendicularmark(t.AC, line(t.VB, H_b), blue, size=10, quarter=3);

        draw(circle(t.VC, H_a, H_b));
        draw(H_a--H_b, grey);
        
        markangle(n = 1, H_a, H, point(t.VC), radius=12, red);
        markangle(n = 1, H_a, H_b, point(t.VC), radius=14, red);

        clipdraw(circle(t.VA, H_a, H_b));

        //draw(arc(circle(t.VA, H_a, H_b), -5, 185));

        markangle(n = 1, H_a, point(t.VB), H_c, radius=16, red);

        draw(circle(H, H_a, H_c), dashed+red+bp*0.7);
    \end{asy}
\end{figure}

\noindent
\begin{minipage}{0.65\linewidth}
    \begin{lemma}\label{lem:concycle}
        Четырехугольник $ABCD$ является вписанным, если $\angle ABC$ равен смежному углу $\angle ADC$.
    \end{lemma}
\end{minipage}
\hspace{0.05\linewidth}
\begin{minipage}{0.3\linewidth}
    \begin{asy}
        size(3.5cm);
        point O = (0, 0);
        circle omega = circle(O, 1); draw(omega);

        point A = angpoint(omega, 210);
        point B = angpoint(omega, 100);
        point C = angpoint(omega, 15);
        point D = angpoint(omega, 320);

        point P = intersectionpoints(circle(B, 2), line(A, B))[0];
        
        draw(P--A--B--C--D--A);
        markangle(n = 1, line(B, A), line(A, D), radius=12, red+dashed);
        markangle(n = 1, B, C, D, radius=14, red);

        markangle(n = 2, D, A, B, radius=14, blue);
    \end{asy}
\end{minipage}
