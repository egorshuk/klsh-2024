\documentclass[12pt]{article}
\usepackage[utf8]{inputenc}
\usepackage[T2A]{fontenc}
\usepackage{amsmath, amsfonts, amssymb, arcs}
\usepackage{geometry, graphicx, hyperref, fancyhdr, enumitem, indentfirst, multicol}
\usepackage[english, russian]{babel}

\geometry{
    a4paper,
    right = 1cm,
    left = 1cm,
    top = 2cm,
    bottom = 2 cm,
    footnotesep = 5mm,
}
\setlength{\headheight}{16pt}
\interfootnotelinepenalty=10000

\hypersetup{
    colorlinks=true,
    linkcolor=red,
    urlcolor=blue,
    pdftitle={Топ-5 теорем, которые не помогут построить дом},
    pdfauthor={Лунёв Егор}
}

\pagestyle{fancy}
\fancyhead{} 
\fancyhead[L]{КЛШ-2024}
\fancyhead[R]{\href{https://t.me/standing_on_my_neck}{\color{red}{$\psi_c$}} $\perp$ \href{https://t.me/egorrshuk}{\color{blue}{$\omega_t$}}}

\fancyfoot{}
\fancyfoot[C]{\thepage}
\setlength{\tabcolsep}{12pt}
\renewcommand{\arraystretch}{1.4}
\setlength{\arrayrulewidth}{0.5mm}


\setlength{\columnsep}{1cm}
\setlength{\columnseprule}{1pt}
\def\columnseprulecolor{\color{blue}}

\newcommand{\task}[2]{\texttt{(#1)} #2}

\let\ch\undefined
\renewcommand{\labelenumii}{(\alph{enumii})}
\counterwithin*{footnote}{page}
\newcommand{\frule}[2]{\noindent \textcolor{red}{\rule{#1\textwidth}{0.5pt}}\textcolor{blue}{\rule{#2\textwidth}{0.5pt}}}
\renewcommand{\footnoterule}{\vspace{1mm}\frule{0.247}{0.153}\vspace{1mm}}
\newcounter{wishlist}
\newcommand{\wishlisted}{\hfill{$\bigcirc$}\stepcounter{wishlist}}

\newlist{tasks}{itemize}{2}

\setlist[tasks, 1]{
    label=$\bigcirc$
}

\usepackage[inline]{asymptote}
\begin{asydef}
    settings.outformat = "pdf";
    import geometry;
\end{asydef}

\begin{document}

    \begin{center}
        
        \Large
        \textbf{Задачки попроще}
        \large
        \vspace{5mm}
        
        НТН
        
        \vspace{5mm}
        
        Лунёв Егор (\href{https://t.me/egorrshuk}{\texttt{@egorrshuk}})

        \vspace{5mm}
    \end{center}
    \begin{multicols*}{2}
        [В данных задачах даны базовые леммы, теоремы, свойства, которые стоит знать и использовать при решении задач по геометрии. Если для вас задачи с моего факультатива слишком сложны -- решайте это.]
    
    \begin{tasks}
        \item \task{Лемма об угле между касательной и хордой}{Описанная окружность треугольника $(ABC)$ касается прямой $BD$. Докажите, что $\angle CBD = \angle CAB$.}

        \begin{center}
            \begin{asy}
                size(8cm, 5cm);
                triangle t = triangleabc(5, 3, 3.5); 
                draw(t); 

                circle omega = circle(t.A, t.B, t.C); draw(omega, linewidth(bp));

                line k = tangent(omega, t.VC); draw(k, blue);
                markangle(n = 1, k, line(t.AC), radius=20, red);

                markangle(t.C, t.B, t.A, radius=20, red);

                draw(box((-1.5, -1), (1, 1)), invisible);
            \end{asy}
        \end{center}

        \item Биссектриса угла $A$ треугольника $ABC$ пересекает его описанную окружность в точке $L$. Докажите, что $BL = CL$.

        \begin{center}
            \begin{asy}
                size(8cm, 5cm);
                triangle t = triangleabc(11.1, 12.3, 11);
                draw(t, linewidth(bp)); 

                circle omega = circle(t.A, t.B, t.C); draw(omega);

                line b = bisector(t.VC); 
                point L = intersectionpoints(b, omega)[0]; dot(L);
                draw(t.C--L); draw(t.A--L, dashed+blue, StickIntervalMarker(1, 1, size=7)); draw(t.B--L, dashed+blue, StickIntervalMarker(1, 1, size=7));

                markangle(n = 1, t.A, t.C, L, radius = 20, red);
                markangle(n = 1, L, t.C, t.B, radius = 25, red);
            \end{asy}
        \end{center}

        \item Докажите, что сумма углов выпуклого $n$-угольника равна $180^\circ (n-2)$.

        \item Докажите,что если $\overset{\frown}{AB}$ и $\overset{\frown}{CD}$ -- меньшие дуги окружности $\omega$, тогда если $P$ -- точка пересечения $AC$ и $BD$, то \[\angle APB = \frac{\overset{\frown}{AB} + \overset{\frown}{CD}}{2}.\]

        \item Докажите, что если биссектрисы углов $B$ и $C$ треугольника $ABC$ пересекаются в точке $I$, то \[ \angle BIC = 90^\circ + \frac 1 2 \angle BAC.\]

        \item Докажите, что вершины $B$ и $C$ треугольника $ABC$ равноудалены от прямой, содержащей медианы $AM$.

        \begin{center}
            \begin{asy}
                size(8cm, 5cm);
                triangle t = triangleabc(9, 11, 10); draw(t, linewidth(bp)); 

                point M = midpoint(t.AB);
                draw(t.A--t.B, StickIntervalMarker(2, 1, size=7));
                line m = line(t.C, M); draw(m);

                point Pa = projection(m)*t.A; 
                point Pb = projection(m)*t.B; 

                draw(Pa--t.A, blue+dashed, StickIntervalMarker(1, 2, size=5)); draw(Pb--t.B, blue+dashed, StickIntervalMarker(1, 2, size=5));

                perpendicularmark(line(t.VA, Pa), m, quarter = 2, red, size=7);
                perpendicularmark(line(t.VB, Pb), m, quarter = 2, red, size=7);

                dot(Pa);
                dot(Pb);
                dot(M);
                draw(box((-1, -2), (0,0)), invisible);
            \end{asy}
        \end{center}
        %TODO : same task

        \item \task{Теорема о биссектрисе}{В треугольнике $ABC$ проведена биссектриса $AL$. Докажите, что \[\frac{BL}{CL} = \frac{AB}{AC}.\]} 

        \item \task{Теорема о внешней биссектрисе}{В треугольнике $ABC$ проведена внешняя биссектриса $AL$. Докажите, что \[\frac{BL}{CL} = \frac{AB}{AC}.\]}
    \end{tasks}
    \end{multicols*}
\end{document}