\documentclass[12pt]{article}
\usepackage[utf8]{inputenc}
\usepackage[T2A]{fontenc}
\usepackage{amsmath, amsfonts, amssymb, arcs}
\usepackage{geometry, graphicx, hyperref, fancyhdr, enumitem, indentfirst, multicol}
\usepackage[english, russian]{babel}

\geometry{
    a4paper,
    right = 1cm,
    left = 1cm,
    top = 2cm,
    bottom = 2 cm,
    footnotesep = 5mm,
}
\setlength{\headheight}{16pt}
\interfootnotelinepenalty=10000

\hypersetup{
    colorlinks=true,
    linkcolor=red,
    urlcolor=blue,
    pdftitle={Топ-5 теорем, которые не помогут построить дом},
    pdfauthor={Лунёв Егор}
}

\pagestyle{fancy}
\fancyhead{} 
\fancyhead[L]{КЛШ-2024}
\fancyhead[R]{\href{https://t.me/standing_on_my_neck}{\color{red}{$\psi_c$}} $\perp$ \href{https://t.me/egorrshuk}{\color{blue}{$\omega_t$}}}

\fancyfoot{}
\fancyfoot[C]{\thepage}
\setlength{\tabcolsep}{12pt}
\renewcommand{\arraystretch}{1.4}
\setlength{\arrayrulewidth}{0.5mm}


\setlength{\columnsep}{1cm}
\setlength{\columnseprule}{1pt}
\def\columnseprulecolor{\color{blue}}

\newcommand{\task}[2]{\texttt{(#1)} #2}

\let\ch\undefined
\renewcommand{\labelenumii}{(\alph{enumii})}
\counterwithin*{footnote}{page}
\newcommand{\frule}[2]{\noindent \textcolor{red}{\rule{#1\textwidth}{0.5pt}}\textcolor{blue}{\rule{#2\textwidth}{0.5pt}}}
\renewcommand{\footnoterule}{\vspace{1mm}\frule{0.247}{0.153}\vspace{1mm}}
\newcounter{wishlist}
\newcommand{\wishlisted}{\hfill{$\bigcirc$}\stepcounter{wishlist}}

\newlist{tasks}{itemize}{2}

\setlist[tasks, 1]{
    label=$\bigcirc$
}

\usepackage[inline]{asymptote}
\begin{asydef}
    settings.outformat = "pdf";
    import geometry;
\end{asydef}

\begin{document}

    \begin{center}
        
        \Large
        \textbf{Ещё задачек попроще}
        \large
        \vspace{5mm}
        
        НТН
        
        \vspace{5mm}
        
        Лунёв Егор (\href{https://t.me/egorrshuk}{\texttt{@egorrshuk}})

        \vspace{5mm}
    \end{center}
    \begin{multicols}{2}
    [
        В прошлый раз довольно большой количество вас решало эти задачи, поэтому получайте еще!
    ]
        \begin{tasks}
            \item Точки $A, B, C, D$ лежат на окружности. Точки $M , N , K, L$ -- середины дуг $AB, BC, CD, DA$ соответственно. Докажите, что $M K \perp N L$.

            \begin{center}
                \begin{asy}
                    size(8cm, 6cm);
                    circle omega = circle(origin(), 1); draw(omega, blue);

                    point A = angpoint(omega, 200);
                    point B = angpoint(omega, 144); 
                    point C = angpoint(omega, 80); 
                    point D = angpoint(omega, -20); 

                    point M = angpoint(omega, 172);
                    point N = angpoint(omega, 114);
                    point K = angpoint(omega, 30);
                    point L = angpoint(omega, -90);

                    draw(A--B--C--D--A);
                    
                    draw(A--M, gray, StickIntervalMarker(1, 1, size=5));
                    draw(B--M, gray, StickIntervalMarker(1, 1, size=5));
                    draw(B--N, gray, StickIntervalMarker(1, 2, size=5));
                    draw(C--N, gray, StickIntervalMarker(1, 2, size=5));
                    draw(C--K, gray, StickIntervalMarker(1, 3, size=5));
                    draw(D--K, gray, StickIntervalMarker(1, 3, size=5));
                    draw(D--L, gray, StickIntervalMarker(1, 4, size=5));
                    draw(A--L, gray, StickIntervalMarker(1, 4, size=5));

                    draw(N--L, dashed+red);
                    draw(M--K, dashed+red);

                    perpendicularmark(line(M, K), line(N, L), blue);
                    
                    dot("$A$", A, dir(200));
                    dot("$B$", B, dir(145));
                    dot("$C$", C, dir(80));
                    dot("$D$", D, dir(-20));

                    dot("$M$", M, dir(172));
                    dot("$N$", N, dir(114));
                    dot("$K$", K, dir(30));
                    dot("$L$", L, dir(-90));
                \end{asy}
            \end{center}

            \item Продолжения биссектрис треугольника $ABC$ пересекают описанную окружность $(ABC)$ в точках $A_1, B_1, C_1$. Докажите, что высоты треугольника $A_1B_1C_1$ лежат на прямых $AA_1, BB_1, CC_1$.

            \begin{center}
                \begin{asy}
                    size(8cm, 6cm);
                    triangle t = triangleabc(7,8,9); draw(t, linewidth(bp)); label(t);

                    circle omega = circle(t.A, t.B, t.C); draw(omega, blue);

                    line ba = bisector(t.VA);
                    line bb = bisector(t.VB);
                    line bc = bisector(t.VC);

                    point A1 = intersectionpoints(omega, ba)[1];
                    point B1 = intersectionpoints(omega, bb)[1];
                    point C1 = intersectionpoints(omega, bc)[0];

                    draw(t.A--A1, gray);
                    draw(t.B--B1, gray);
                    draw(t.C--C1, gray);

                    markangle(n=1, A1, t.A, t.C, radius = 25, blue); 
                    markangle(n=1, t.B, t.A, A1, radius = 20, blue); 

                    markangle(n=2, B1, t.B, t.A, radius = 25, blue); 
                    markangle(n=2, t.C, t.B, B1, radius = 20, blue); 

                    markangle(n=3, C1, t.C, t.B, radius = 25, blue); 
                    markangle(n=3, t.A, t.C, C1, radius = 20, blue); 

                    draw(A1--B1--C1--A1, red+dashed);

                    perpendicularmark(line(A1, B1), bc, size=8, blue);
                    perpendicularmark(line(B1, C1), ba, size=8, blue);
                    perpendicularmark(line(A1, C1), bb, quarter=4, size=8, blue);


                    dot("$A_1$", A1, dir(20));
                    dot("$B_1$", B1, dir(130));
                    dot("$C_1$", C1, dir(-90));
                \end{asy}
            \end{center}


            \item В прямоугольном треугольнике $ABC$ с прямым углом при вершине $C$ катет $BC$ равен $a$, радиус вписанной окружности равен $r$. Вписанная окружность касается катета $AC$ в точке $D$. Найдите хорду, соединяющую точки пересечения окружности с прямой $BD$.

            \item Диагонали $AC$ и $BD$ вписанного четырехугольника $ABCD$ взаимно перпендикулярны и пересекаются в точке $P$. Докажите, что прямая, проходящая через точку $P$ и середину стороны $AD$, перпендикулярна $BC$.

            \begin{center}
                \begin{asy}
                    size(8cm, 6cm);

                    circle omega = circle(origin(), 1); draw(omega, blue);
                    
                    point A = angpoint(omega, 160);
                    point B = angpoint(omega, 100);
                    point C = angpoint(omega, 20);
                    point D = angpoint(omega, -100);

                    draw(A--B--C--D);
                    draw(A--C, gray);
                    draw(B--D, gray);

                    point P = intersectionpoint(line(A,C), line(B, D)); 

                    perpendicularmark(line(A, C), line(B, D), red);


                    point M = (A+D)/2; draw(A--D, StickIntervalMarker(2, 1, size=7));

                    draw(line(M, P), dashed+blue);

                    perpendicularmark(line(M, P), line(B,C), quarter=3, red);

                    

                    dot(M);
                    dot("$P$", P, dir(-55));
                    dot("$A$", A, dir(160));
                    dot("$B$", B, dir(100));
                    dot("$C$", C, dir(20));
                    dot("$D$", D, dir(-100));
                \end{asy}
            \end{center}

            \item Из точки $A$ проведены два луча, пересекающие данную окружность: один -- в точках $B$ и $C$, другой -- в точках $D$ и $E$. Известно, что $AB = 7$, $BC = 7$, $AD = 10$. Найдите $DE$.

            \item На сторонах $AB$ и $CD$ квадрата $ABCD$ выбраны точки $P$ и $Q$ соответственно. Далее, на прямой $PQ$ выбрали точку $R$. Окружности $(APR)$ и $(CQR)$ вторично пересекаются в точке $T$. Докажите, что $T$ лежит на диагонале квадрата.
        \end{tasks}
    \end{multicols}
\end{document}